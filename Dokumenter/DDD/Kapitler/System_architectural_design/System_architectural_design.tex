\chapter{System Architectural Design}

\section{System Components}



This chapter seeks to identify the system components. It provides a block definition diagram (bdd) of the system, where the system components are determined along with the static relationship between them. The block definition diagram is shown in figure \ref{fig:block_diagram}:

\begin{figure}[H]
\centering
\includegraphics[width=0.95\textwidth]
{billeder/bdd_overordnet.pdf}
\caption{Block diagram of the system.}
\label{fig:block_diagram}
\end{figure}

The diagram consists of system-blocks along with parts associated to each block. The system-blocks are depicted as two-compartment blocks with the name of the block in the first compartment, and sub parts in the second compartment. In the next section, a short description of each system-block is given.

\subsection{Component description}
\begin{enumerate}
\item[•] \textbf{PC:} This block constitutes the machine in the head quarter (HQ) on which the COP-software will be executed. It also has a GPS module, so that the location of the HQ is always known. The PC has a telecommunication module, in order to be able to communicate with the rest of the system.
\item[•] \textbf{Server:} The server will facilitate communication between the other blocks. In addition, it will store user information along with logs locally in an internal database.
\item[•] \textbf{Dismounted COP:} This block constitutes the machine on which the condensed COP-software will be executed. The dismounted COP will be used by the dismounted users in the field. It has a GPS module, so that the location of the dismounted users is always known. Furthermore it has a telecommunication module so that it will be able to communicate with the rest of the system. 
\end{enumerate}
\section{Concept of Execution}

\section{Interface Design}
This section seeks to describe the interface characteristics of the system components. It provides an internal block diagram (bdd) of the system, where the interfaces of the system components are identified, as long as the external interfaces of the system. The internal block diagram of the overall system is shown in figure \ref{fig:internal_block_diagram}:

\begin{figure}[H]
\centering
\includegraphics[width=0.95\textwidth]
{billeder/ibd_overordnet.pdf}
\caption{Internal block diagram of the system.}
\label{fig:internal_block_diagram}
\end{figure}

\subsection{PC}
In this section the internal interfaces of the PC to be used in this system are specified in greater detail. All the sub parts of the PC are connected to the processor which manages all logic operations and functions, while the telecommunication module enables the PC to communicate with the remaining system components. It is not within the scope of this project to develop the PC itself, however the COP is to be executed on the PC. Therefore the interfaces of the PC are identified, to ensure that the PC - and thereby the COP - can communicate with the rest of the system. The internal block diagram of the PC is shown in figure \ref{fig:internal_diagram_PC}:

\begin{figure}[H]
\centering
\includegraphics[width=0.95\textwidth]
{billeder/ibd_PC.pdf}
\caption{Internal block diagram of the PC.}
\label{fig:internal_diagram_PC}
\end{figure}
